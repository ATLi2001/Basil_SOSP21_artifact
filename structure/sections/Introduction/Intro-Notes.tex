\subsection{Intro-Notes}

\subsection{Header}

 - This paper introduces Indicus\\
 - interactive, scalable DB in the byz fault model\\
 - first to be fully client driven\\
 \nc{this might be personal taste, but I don't think being
 client-driven is particular novel or important. Being leaderless
 on the other hand seems like something we should emphasise}
 - therefore: fair and can scale to replica bandwidth/processing limits
 
 \subsection{Context}
 - people/entities would like to share data and/or split replication costs. Keep joint records of relevant data (i.e. for cross Audit): Imagine a transaction clearance between banks  (Libra as example: consortium of parties to manage a currency), a supply chain, different medical providers sharing patient data (for easy migration) - any system where users would like to easily migrate between providers.
 - Parties no longer from the same authority, potentially untrusted. --> byzantine fault tolerance not for increased software failure resistance (bugs etc), but to tolerate some level of misbehavior
 - While BFT was not warranted in single entity replication (extra cost for some more FT), it should be the standard for heterogenous replication. This is the basis for Blockchain where BFT is the norm.\\
\nc{I don't think we need to be defensive about BFT. It's the norm
in blockchain and blockchain is "hot"}
 -Alternatively, big corporations that use transactional systems that are geo-replicated might be worried about administrative domains: I.e. google spanner having replicas located in authoritarian regimes or out of its own control. does not "trust" its "own" replicas.
 
 -
 - SMR powerful fault tolerant technique. Creates illusion of single Database, thus simplifying application design. Replication is masked.
 \nc{this comes out of no where} \\
 
- ACID transactions: Absolute data integrity. Simplifies concurrency control. Intuitive data access logic.
Concurrent access to retail inventories,  bank balance, stocks is unavoidable. I.e. serializable isolation makes reasoning simple for application developers.

 Interactive TX: Most general for any kind of application.
\fs{ Any system that provides Interactive TX can handle  One shot for example.}\\
 - Goal: Make BFT accessible to the mainstream. Requires scalability and low latency. Cost needs to be tolerable - more reasonable if replication is done as consortium (each party pays for subset).
\nc{I don't think we particularly need to discuss the benefits of transactions or the benefits of interactive transactions yet. I think it's cleaner to say that blockchain, backed up by BFT, has the problem of being totally ordered. Moves the text more quickly.}

\subsection{Existing systems shortcomings}
- systems were designed for different assumptions: single authority low replication (historical use cases: aviation, space, nuclear power) vs high replication degree in a consoritum of untrusted parties\\
- non-interactive transactions\\
- totally ordered\\
- leader based: fairness and scalability bottleneck \\
- Crash Failure model\\
- single sharded\\

\subsection{Key insights}
- ?? \\
- mismatch between requirements of a transactional store and implementation of totally ordered log\\
- implicit partial order suffices for TX 
\\
-EVERYTHING is a partial order: Entire Life cycle is parallel if non-conflicting\\

- flip the problem of trying to add scalability to existing BFT systems. add BFT to efficient CF DBs\\

- put clients in charge: responsible for their own liveness;  a natural way to scale a system\\
\nc{is this the "standard" way to scale a system?}

- do single slot binary consensus\\
-
\paragraph{Challenge} Empowering byzantine Clients is complicated and dangerous

\subsection{Positive Implications of Insight/Design}
- robust to censorship and frontrunning (if network not adversarial) as there exist no central authority \\
-  allows commutative/non-conflicting Transactions to both be executed and validated out of order, thus maximizing parallelism. \\
-  minimizes the state, communication and computation load on replicas, as Clients serve as both execution hub and broadcast channel for their own Transactions. \\
\nc{I don't think everyone would agree that this is a good thing,
replicas are big beefy machines at the datacenter, whereas clients
might be lightweight devices where you explicitly don't want to run much computation}
- As any Quorum system, Indicus is inherently load-balanced as there exists no leader bottleneck and all replicas have equal responsibility\\
- In Indicus liveness is a client local property. Unlike SMR, where the entire system halts during view changes, Byzantine participants may stall system progress only for the objects their transactions touch. Hence, the system appears live to any non-conflicting Transaction. View changes are parallel to one another and are only triggered when a participant is interested, which is inline with our "insight" that everything is a partial order.

\subsection{Results discussion}
- latency should not be that much higher compared to Tapir\\
- throughput better than an atomic broadcast version\\
- abort rate doesnt rise too fast; mvtso improves abort rater over OCC (if not, then its useless overhead and we can use OCC)


\subsection{Limitations}
 - Like any speculative system, Indicus is vulnerable to high contention in which case it needs to resort to aborts\\
 - This can especially be exploited by byzantine participants if the network is adversarial.\\
 - 

\subsection{Contribution summary}
 - Present definitions for what it means to provide ACID (specifically Isolation) guarantees in a byzantine setting\\
 - We present design, implementation and evaluation of Indicus, a fully client-driven BFT ACID transactional system \\
 - Design of a leaderless, client-driven execution and replication protocol that does not enforce a total order\\
 - Present 2 flavors of Indicus: Indicus3 and Indicus5 that explore the trade off between replication degree and performance (latency and computational overhead)
 - The design of a more aggressive optimistic concurrency control scheme that is robust to byzantine behaviors