


This paper introduces Indicus, the first leaderless transactional key-value store that is robust in the Byzantine Fault Model. Indicus aims to mitigate the tension between real-world, highly commutative transaction workloads striving for scalability, and the simplicity that totally ordered ledger abstractions such as Blockchains or State Machine Replication offer. Specifically, this paper asks the question: how can we enable \textit{mutually distrustful parties} to consistently and reliably share data, while minimizing centralization. \fs{this maybe sounds too much like a permissioneless blockchain}

The ability to share data online offers exciting opportunities, however, increased datasharing also raises the concern of how to \textit{decentralize trust}. In banking, systems like SWIFT (cite) enable financial institutions to quickly and accurately enable cross-institutional transaction clearance, at the cost of placing their trust in the centralized SWIFT network. In manufacturing, online data sharing can improve accountability and auditing amongst the globally distributed supply chain, but there may not be an identifiable source of trust. Consider the supply chain for the latest iPhone: it spans three continents, and hundreds of different contractors \cite{AppleSup} that may neither trust Apple, nor each other, yet must be willing to share and agree on information concerning the construction of the same product. Moreover, single authorities that distribute their dataceneters globally (i.e. Google, Amazon), may not even trust their \textit{own} datacenters located in authoritative domains or legislations out of its control.

Recognizing this challenge by both the research and industrial communities, much effort has focused on enabling shared computation between mutually distrustful parties in the context of Byzantine Fault Tolerance (BFT) and Blockchains. Systems proposed in the literature of BFT 
