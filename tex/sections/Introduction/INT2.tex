This paper presents the design and evaluation of \sys{}, the
first leaderless and scalable transactional key-value store that is robust to byzantine faults.

Safe and efficient online data sharing among mutually distrustful
parties offers exciting opportunities for a variety of applications,
including healthcare~\cite{}, financial services~\cite{}, supply chain
management~~\cite{} and more~\cite{}. \nc{I understand that you want to move 
to the main point quickly but I think that one example might be useful here still, just
one}. Byzantine fault tolerant (BFT)
systems~\cite{castro1999practical,martin2006fast,kotla2007zyzzyva,  gueta2018sbft,clement2009making,buchman2016tendermint,yin2019hotstuff,Clement09Upright,duan2014hbft, pires2018generalized,bessani2014state,lamport2011byzantizing,arun2019ezbft, malkhi2019flexible,duan2014hbft,yin2003separating, Guerraoui08Next, Kotla04High,liskov2010viewstamped} and permissioned blockchains~\cite{Hyperledger,EthereumQuorum, buchman2016tendermint, al2017chainspace,kokoris2018omniledger,gilad2017algorand, baudet2019state} are at the center of these new services. These protocols are guaranteed to
produce the same totally ordered log of operations across mutually distrustful
participants. 

Maintaining a totally ordered log of requests, despite its appealing simplicity, is problematic as such an approach is hard to scale and costly to implement. This total order is in fact often unnecessary; modern distributed applications primarily consist of independent operations that could be executed concurrently. Supply chains for instance, despite their name, are actually complex networks that generate and process logically concurrent transactions.

Existing research attempts to mitigate the aforementioned scalability bottleneck through a combination of sharding, cross-shard atomic commit protocols and centralised ordering services~\cite{kokoris2018omniledger,al2017chainspace,padilha2016callinicos}. These systems suffer from three primary drawbacks \one ~ they limit the expressivity of the transactions they support \two ~ they introduce unnecessary coordination across replicas and finally,
\three ~ they often centralise the ordering of requests at a single leader, which raises fairness concerns. These limitations all stem from the same fallacy: the mistaken belief that building out a total order of operations is necessary for correctness.
 
In reality, building \textit{the abstraction} of a total order is sufficient, and distributed databases have long recognised this fact~\cite{crooks2018obladi,bernstein1979fas,Papadimitriou1979serializability,adya99weakconsis}. Serializability, the gold-standard database criterion, defines a correct execution as one that is \textit{equivalent} to a correct schedule~\cite{bernstein1979fas,Papadimitriou1979serializability,bernstein1979fas}. This definition allows concurrent operations to execute in parallel and orders conflicting operations only.  Byzantine systems need be no different. This paper consequently argues for a paradigm "flip". The
right way to build BFT data-sharing systems is not to layer database-like transactions and sharding on top of a BFT log, but instead to do the opposite and build out the abstraction of a BFT log atop a partially ordered distributed database.

To this effect, we design \sys{}, the first leaderless and scalable key-value store that is resilient against
Byzantine faults. We specifically formalise the notion of \textit{byzantine isolation}, the correctness
guarantee that any BFT distributed database should enforce.  \sys{} provides three main benefits.

 First, it overcomes the scalability bottleneck of a low-level
implementation based on a totally ordered BFT log. \sys instead leverages databases' ability to support highly concurrent transaction
processing while maintaining the abstraction
of a serial execution.

% an explicit ordering service 

Second, by doing away with an underlying totally ordered log, \sys{} can
sidestep the growing concerns~\cite{} about whether blockchains
order transactions fairly. Though the notion of fairness in this
context has not been rigorously defined, leaving the ordering to the
whim of a potentially Byzantine leader, as some protocols do~\cite{
Kotla07Zyzzyva,castro1999practical}
is intuitively problematic \cite{herlihy2016enhancing}. In contrast, \sys leaves the
responsibility of driving the replication and distributed commit of a
transaction to the client that proposes it \changebars{fs: cut imo}{and allows separate 
shards to order transactions independently as the total order abstraction afforded by byzantine serializability can be maintained. }  \fs{Limitation of client computation can come here}

Third, \sys{} avoids the duplicate ordering costs of transactional systems
built on top of state machine replication. As Tapir noted~\cite{zhang2015tapir,mu2016consolidating},
these systems require a consistent ordering of
operations within each shard for replication, and additionally enforce a serial order of transactions across
shards for distributed commit. In contrast, \sys{} integrates distributed commit with 
replication and
limits coordination to the validation step of the distributed commit
protocol.

In practice, reaping these benefits requires us to address several
technical obstacles. As in any OCC-based protocol, \sys
is vulnerable to aborts if transactions interleave unfavorably during
validation---but these concerns are compounded in a Byzantine
setting. First, since reading from a single local replica can no
longer guarantee integrity, it becomes necessary to read remotely,
thus increasing the window of time during which \changebars{unfavorable}{unfortunate}
interleavings may occur. Second, Byzantine clients and replicas may
collude to actively sabotage the commit chances of transactions issued
by correct clients. 

Byzantine behaviour in effect introduces a tension that is not present in non-byzantine
distributed databases: optimistic and aggressive concurrency control mechanisms known
to improve performance (~\cite{kung1981occ,bernstein1983mcc,reed1983atomic,xie2015callas,zhang2015tapir}) in failure-free executions also increase the system's vulnerability to byzantine faults. Consider for instance runtime pipelining~\cite{xie2015callas,su2017tebaldi} or multiversioned timestamp ordering (MVTSO)~\cite{bernstein1983mcc,reed1983atomic}: both allow writes to become visible to other operations before a transaction commits. While early write visibility helps reduce abort rates for
contended workload, it can cause transactions to stall on uncommitted operations whose writes they have observed. To mitigate this tension, we design \sys{} to follow the ethos of \textit{independent operability}: both safety (serializability) and liveness (the ability to issue and complete transactions) are \textit{local} properties. They remain, at all times, per client and per-object. \mb{I am confused by the use of ``local.'' Are we using it the same
as Herlihy and Wing from the Linearizability paper? If not, I'd suggest we come
up with a different term to avoid confusion. For example, Serializabiliy is not
a local property of a system comprised of multiple objects (it is not the case
that the system is Serializable if and only if each object is Serializable).}Specifically, we develop a variant of multiversion timestamp ordering that \changebars{limits byzantine clients ability to strategically abort honest clients' transactions}{ precludes byzantine clients from
aborting honest clients' transactions}. \fs{This is currently too strong. Can we be more clear about how our MVTSO achieves this? Only thing that comes to mind is that writes are delayed and do not have external effect early, and dependencies are only on f+1 matching reads}
 We additionally develop a novel \textit{fallback} mechanism that allows clients to finish others' pending transactions when stalled. Importantly, this fallback mechanism is per-object, and can thus take place concurrently with other non-conflicting operations. This is in contrast with traditional BFT view-changes which fully preclude normal operation processing.  Finally, we remain, at all times, fully leaderless. \fs{not true for the fallback, where we intentionally elect a leader. The difference is, that in Indicus each Tx has its own leader (which is usually the client, but can become a replica). We do not want to mislead.}

Naturally, \sys{} also has limitations. First, it shifts some
responsibilities from replicas to clients, which increases a client's computation load. Second, \sys{} is optimistic: it greedily allows transactions to read uncommitted data and allows replicas to process operations out of order. This approach can lead to high abort rates under heavy contention.
\nc{while i think we should include
limitations because that's a good thing to do, it's not obvious to me that the above paragraph is useful. Those limitations seem kinda obvious?}
\fs{agree, the interleaving vulnerability part is also mentioned already. The client cost can be moved up - see earlier comment}

In summary, we make the following three contributions: 
\begin{itemize}
\item We introduce the notion of Byzantine Isolation and prove that \sys guarantees
Byzantine Serializability.
\item We introduce a novel concurrency control and recovery mechanism that balance the need for high-throughput in
the common case with resilience to Byzantine attacks.
\item Our system remains leaderless and exhibits linear communication complexity, unlike prior BFT systems.
\end{itemize}


The rest of the paper is structured as follows. Section~\ref{sec:model} formalises \sys{}'s correctness guarantees.
Section~\ref{sec:overview} outlines
\sys's architecture; we summarise our concurrency control control protocols and recovery protocols in \S\ref{sec:cc} and \S\ref{sec:fallback}. We
prove correctness in \S\ref{sec::fallback} and evaluate \sys in \S\ref{sec:}. Finally, we summarise related work (\S\ref{sec:rel}) and conclude (\S\ref{sec:conc}).

