%-------------------------------------------------------------------------------
\section{Related Work/Comparison}  
%-------------------------------------------------------------------------------
A lot of recent effort has gone into designing high throughput and low latency databases that leverage synergies between transaction and replication layer to squeeze out any last performance. The recently proposed TAPIR transaction protocol leverages redundancy between transaction ordering and replication ordering to reduce total roundtrips, thus reducing latency in wide area networks. TAPIR shares several similarities with our system, most notably the absence of a leader and the resulting unordered validation structure. TAPIR too, leverages optimistic concurrency control to allow for concurrency among commutative transactions. When congestion is high however, throughput and tail latency worsen as abort rate grows. Janus avoids transaction aborts by dynamically re-ordering conflicting transactions. This however, is made possible by assuming one-shot transactions, i.e. fixed read/write keys, and thus reduces the application generality. Another DB, Carousel similarly assumes a restricted Transaction model in order to parallelize execution and validation. While all of these databases offer low latency replication, they are fundamentally limited to tolerating crash-failures. This strong failure assumption makes them less secure and not suitable for storing mission-critical, financial or highly sensitive data. 
With the surge in Blockchain interest Byzantine Fault Tolerant (BFT) protocols are experiencing a second Spring. While originally developed to tolerate arbitrary bugs, these protocols find increasing importance in settings where participants are untrusted or malicious. Permissioned Blockchains, organized by a consortium of registered participants can use traditional BFT State Machine Replication (SMR) protocols in order to achieve agreement. Starting with PBFT, there have been numerous adaptations such as FaB or Generalized byzantine Paxos, and most notably Zyzzyva which leverages speculative execution and a semi-client driven protocol to reduce latency. SBFT modifies PBFT to scale to large replication degrees by utilizing collectors, threshold signatures and a fast path akin to Zyzzyva. Aardvark states the importance of robustness against byzantine failures and takes measures to increase the rate of leader rotation. Tendermint pushes this idea to the extreme by rotating leaders continuously, at the cost of assuming a synchronous network. HotStuff too, explores the use of rotating leaders by exploiting the symmetry in PBFTs phases. 
Nevertheless, all protocols derived from the PBFT family suffer from the leader bottleneck as well as enforcing a total order even on commutative operations. BFT-Mir claims the leader proposing speed as the practical bottleneck in most implementations and improves upon this by allowing all replicas to act as proposers.
Biblos achieves leaderless SMR by leveraging a non-skipping timestamp protocol. It furthermore allows for commutative Transactions to be executed in parallel at the cost of requiring read/write key sets to be known in advance. In order to preserve liveness however Bilbos falls back to a PBFT resolution. \fs{Uses all to all, predefined TX, pbft fallback, timestamp phase and proofs required}
 Q/U too, offers leaderless agreement via Qurorums for a limited read/write interface, but fails to terminate under contention. H/Q improves upon Q/U by adding PBFT fallback path under contention. Liskov et Al further explore byzantine Quorum protocols for a Read/Write interface, giving special attention to bounding the effects of byzantine clients. 
While the literature on BFT state machine replication is extensive, the efforts to offer a transactional interface for BFT is scarce. SMR itself can be utilized as a straw-man system to implement pre-defined Transactions (one-shot or stored procedures) by enforcing a common total order  in which replicas will execute the transactions. 
HRDB offers a dedicated BFT Database, but assumes a trusted shepheard layer, thus not being truly BF resilient
Byantium offers Snapshot Isolation for Transactions by re-purposing PBFT as atomic broadcast. It executes requests only at a primary and uses replicas to validate results in the total order defined by the SMR protocol. Augustus implements mini-transactions, a limited TX model that declares all operations before execution. It offers scalability via partitioning and achieves consistency within partitions by utilizing atomic broadcast. Whiile Augustus assumes an optimistic execution model that allows for aborts under concurrency, its follow up work Callinicos implements a locking scheme in order to avoid Transaction aborts. To do so efficiently it resorts to a limited transaction model that requires knowledge of read/write sets and otherwise locks an entire partition in order to guarantee mutual exclusion, thus voiding any concurrency. BFT Deferred update replication adopts an interactive OCC transaction model comparable to ours, allowing execution to be speculative at clients. However, it uses PBFT atomic broadcast to enforce SMR on validation, thus resorting to a leader, and totally ordering all requests. It moreover does not extend to multiple shards.
Chainspace and Omniledger implement blockchain sharding by layering 2PC and atomic broadcast (within shards) for UTXO transactions. Rapidchain offers efficient sharding for a permissionless system (what else? didnt read much, not so relevant). Hyperledger (uses simulation-ordering-validation architecture. uses Raft for trusted ordering).. Mention other Permissioned Blockchain systems: Ethereum Quorum? (uses Raft or IstanbulBFT = implementation of pbft)


\fs{CURP is a CF replication using commutativity too: Makes request durable first and orders later.}

\fs{Mention DAGs for permissionless systems.}

\fs{Databasify paper: tries to add some bandaid db techniques to hyperledger. Transaction re-ordering (still total order), and early aborts.}

\cite{kotla2007zyzzyva}
