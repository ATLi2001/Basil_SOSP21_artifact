\subsubsection{System properties}
We offer Clients an interactive transaction interface that implements ACID transactions. While both Atomicity and Isolation guarantees may be violated by individual byzantine Replicas, Indicus maintains the view of an ACID compliant state to all honest Clients. 

Specifically, we guarantee to Clients that the Database is \textit{byzantine Serializable} as defined below. Intuitively this Isolation level guarantees that all honest Clients experience the Database as if there serializable. In order to formally capture this we lay some ground work: \footnote{Defs for commands etc in accordance with BFT DUr. cite}\\

Let $Op =  \{r, w\} \times K \times V $ and $Dec = \{Commit, \,Abort\}$ be the sets of possible operations and decisions respectively, where $K$ is the set of existing data items (keys) and $V$ the range of possible values. A \textit{request} $req \in (Op \cup Dec) \times C$ maps any such operation or decision to the issuing Client from set $C$. We denote with $Hon \subseteq C$ the subset of honest Clients. 

\textbf{History H.} Informally, a \textit{H} contains the operations (read/write) and decisions (commit/abort) of every transaction issued in the system. Formally, we define a \textit{history H} as a finite sequence of requests.

We define a projection $H|_c$ as the subsequence of requests in $H$ that were issued by Client $c$. A sequence of requests $s = req_i \dots req_{i+t}$ in $H|_c$ form a \textit{Transaction} if $req_i$ is the first request by Client $c$ or $req_{i-1} \in Dec \times c$, and if $req_{i+t} \in Dec \times c$.

We further define:

\textbf{Honest History H(P).} Given protocol P, A \textit{history H} is \textit{honest} if it was generated by participants who all follow P. I.e. concretely, $H(P) \equiv H = H|_{Hon}$.
and

\textbf{Honest-View Equivalent.} A \textit{history H} is honest-view equivalent to a $history H'$ if the Operations and Decisions of all honest Clients are the same and if the final writes are the same.

\textbf{Byz-I} Given a protocol $P$ and an isolation level $I$:
A history H is \textit{byzantine-I} if there exists an honest history \textit{H'} such that H is honest-view equivalent to H' and H' satisfies I. \\

This definition captures the requirements for any byzantine tolerant protocol that strives to maintain byzantine Isolation level I.
Informally a byzantine Isolation level states that the state that honest Clients experience must be explicable by an execution in which all participants were honest. Note, that we make no assumptions on the state a byzantine client \textit{chooses} to experience; Byzantine client reads may be arbitrary, i.e. integrity and legality of both read values and versions is not maintained. \\
We further define byzantine Atomicity. Intuitively, only honest client transactions are guaranteed to experience Atomicity, i.e. all of its operations succeed, or fail jointly. It follows straightforward:\\
\textbf{Byz-A} Given a protocol $P$, a history $H$ and a set of honest client $Hon$. All \textit{Transactions} in $H|_{Hon}$ experience Atomicity.\\

Indicus maintains both \textit{Byz-I} for Isolation level Serializable and \textit{Byz-A}. More pragmatically, all honest clients experience the ACID properties, whereas byzantine clients may \textbf{choose} whether to experience Atomicity and Isolation for their transactions.\\

Next, we define an ideal progress property to limit the influence Byzantine Clients have on system throughput. \\

\fs{this paragraph might perhaps have to go somewhere later, after the protocol is explained and we discuss limitations}
\textbf{Byzantine Independence}
Given a protocol $P$ and an honest Client $c$. The result of a request $r$ issued by $c$ cannot be determinsistically decided by byzantine participants.  \\

If the network is controlled by an adversary, this property is unattainable for Indicus. 
In order to offer this property we must strengthen our assumption on the network. Concretely, while the network may be asynchronous, the adversary does not control the network, and hence, may not reliably impact results. \\
An exception to this are wide-ranged flooding attacks (ddos) which are beyond the scope of this work.
We point out, that a strawman system offering interactive transactions and speculative execution while relying on Atomic Broadcast for Validation ordering suffers the same fallacy: A byzantine leader may always frontrun requests. In fact, even with strengthened network assumptions, such a a system could not offer Byzantine Independence.\\
%We point out, that in fact no system that speculates on execution, and hence may incur aborts, can enjoy \textit{Byzantine Independence}. Not true since in theory one could imagine a scheme that uses "committment certification", but this is impractical


We adapt and define gracious and uncivil executions based on cite(aardvark) to match our model. (i.e. network not sync either).
\textbf{Gracious Execution}
An execution is gracious iff (a) the execution is synchronous with some
implementation-dependent short bound on message delay (b) all clients and servers behave correctly and (c) there is no contention on the objects relevant to the execution.
\textbf{Uncivil}
An execution is uncivil iff (a) there is no bound on message delay (asynchrony) and (b) up to f servers and an arbitrary number of clients are Byzantine 


\subsection{Assumptions on Behaviors}
\subsubsection{Notes}
Clients can do what they want, but we assume there is access control to bound the reach. We dont expect clients to explicitly try to circumvent others progress (this requires knowledge of what they are doing in the first place).
We expect a level of performance, if you dont meet it you are not fit for the system. (Slow behavior is misbehavior)
Replicas can enforce such access lists, i.e. via timeouts of blacklists.
unlikely that a Client can lock out a lot of the system, would mean he has total access control.

Assume byz is rare and have disincentives. We dont expect strategic congestion aborts. Because a) those require knowledge of the keys, b) requires access control and c) requires to post legitimate transactions that will be recorded, might commit since its up to network (imagine a bank account: withdrawing probably illegal, only paying allowed by arbitrary party. now i am paying in order to avoid the real user being able to withdraw)
Have to design protocol so that there is minimal impact of malicious actions.

Since a client cannot drive the system into inconsistency: he can only slow down or cause aborts.
