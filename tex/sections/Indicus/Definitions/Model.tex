%-------------------------------------------------------------------------------
\section{Model and Definitions}
%-------------------------------------------------------------------------------


\subsection{System Model}

We adopt the standard adversarial assumptions of BFT replication. We assume that an arbitrary but finite number of \sys's client are faulty, and that, withing each shard, the number of faulty replicas does not exceed a threshold. Faulty clients and replicas may  deviate arbitrarily from their correct sepcification;  a strong but static adversary can coordinate their actions but cannot  break cryptographic primitivives  like collision-resistant hashes, encryption, and signatures. We denote a message $m$, signed by principal $p$ as $\langle m \rangle_p$.

\sys{}  does not rely on network synchrony for safety; however, it can only guarantee liveness when messages exchanged between correct parties are delayed by no more that a fixed, but potentially unknown time interval~\cite{fischer1985impossibility}.

% urther, \sys{] guarantees liveness one a per-client basis:
% Unlike traditional State Machine Replication protocols, where client liveness  in which the liveness of all Clients is correlated with the fate of the system (or often more specifically a leader), our system guarantees liveness not on a system basis, but on a per client basis. Concretely, we only guarantee liveness to clients that follow the protocol. Conversely, an honest client only loses liveness when it intertwines its fate with byzantine clients.\\

The influence of Byzantine clients may be reduced by enforcing authentication and access control; though it is impossible to prevent authenticated Byzantine clients from compromising the database's integrity, their actions are  retraceable by auditing transaction logs.
